% !TeX TS-program = xelatex

\documentclass{resume}
\ResumeName{王佳顺}

% 如果想插入照片,请使用以下两个库。
% \usepackage{graphicx}
% \usepackage{tikz}

\begin{document}

\ResumeContacts{
  (+86)173-2577-2985,%
  \ResumeUrl{mailto:wjs@hrbeu.edu.cn}{wjs@hrbeu.edu.cn},%
  住址:山东省青岛市崂山区\\
  性别:男,
  出生年月:1998年10月,
  籍贯:江苏兴化,
  民族:汉族,
  政治面貌:中共党员,
  学历:硕士研究生
  %\ResumeUrl{https://blog.fkynjyq.com}{blog.fkynjyq.com} \footnote{下划线内容包含超链接。},%
  %\ResumeUrl{https://github.com/fky2015}{github.com/fky2015}%
}

% 如果想插入照片,请取消此代码的注释。
% 但是默认不推荐插入照片,因为这不是简历的重点。
% 如果默认的照片插入格式不能满足你的需求,你可以尝试调整照片的大小,或者使用其他的插入照片的方法。
% 不然,也可以先渲染 PDF 简历,然后用其他工具在 PDF 上叠加照片。
% \begin{tikzpicture}[remember picture, overlay]
%   \node [anchor=north east, inner sep=1cm]  at (current page.north east) 
%      {\includegraphics[width=2cm]{image.png}};
% \end{tikzpicture}

\ResumeTitle


\section{教育经历}
\ResumeItem
[哈尔滨工程大学|硕士研究生]
{哈尔滨工程大学}
[\textnormal{海洋科学,水声工程学院|}  学术型硕士研究生]
[2021.09—2024.06]

\textbf{GPA: 3.2/4.0(专业排名 1/9)},于青岛创新发展基地接受课程学习,与自然资源部第一海洋研究所联合培养,在中国工程院方国洪院士组建的“大洋环流及潮波动力学团队”进行科研工作,研究方向为\textbf{物理海洋学}。在学术情报分析和海洋数据处理方面积累了大量的研究和实践经验。获学业奖学金多次。

%主要研究成果为:发表在 XXX 期刊上的论文《XXX》。

\ResumeItem
[广东海洋大学|本科生]
{广东海洋大学}
[\textnormal{海洋科学,海洋与气象学院|} 理学学士]
[2017.09—2021.06]

\textbf{GPA: 3.6/5.0(专业排名9/93)},\textbf{物理海洋学}培养方向,在校期间获学业奖学金多次,担任过班级班长及学习委员,在校期间加入中国共产党。担任过校内蓝丝带海洋保护协会办公室部长,参与组建气象领域知名公众号“新云天气象”。






\section{实习经历}

\ResumeItem{自然资源部第一海洋研究所(山东青岛)}
[联合培养研究生]
[2022.09—2024.06] 

\begin{itemize}
  \item \textbf{完成学位论文。}顺利完成硕士学位论文《印度尼西亚海域海洋热浪变化特征及动力分析》,在此期间积累了大量云计算和数据处理分析经验。
    \item \textbf{科研辅助工作。}帮助团队成员处理棘手的技术工作。如对团队科研成果进行核对,并对文献引用情况进行统计计量;帮助对上世纪的纸质版海洋历史数据进行电子化;帮助对全球Argo海洋浮标数据进行自动化筛选。
\end{itemize}

\ResumeItem{国家海洋局湛江海洋环境监测站(广东湛江)}
[数据质量监测实习生]
[2021.03—2021.04] 

\begin{itemize}
  \item \textbf{参与海洋监测站的数据质量监测。}通过监测海洋监测站上传数据,对异常数据进行剔除,保证珠海中心站的数据质量。
  \item \textbf{规章制度和技术手册学习。}通过学习监测站历年年报和国家海洋局系统内部规定,对海洋领域相关的国标,技术规范有充分了解。
\end{itemize}

\ResumeItem{中国人寿保险股份有限公司兴化支公司(江苏兴化)}
[销售实习生]
[2017.07—2017.08] 

\begin{itemize}
  \item \textbf{参与保险业销售新人培训。}通过了保险公司新人培训,掌握了保险产品的相关知识及相关法律。
\end{itemize}

\section{项目经历}

\ResumeItem{\textbf{Extracurricular Physical Oceanography}  }
[ 课程情报项目]
[2021.09 — 2023.04]
\begin{itemize}
  \item 旨在通过引入国际课程强化硕士生第一年的课程学习,使课程学习更好地服务于科研实战
  \item 实现了对全球多所高校的物理海洋学及海洋气象学课程的收集整理
  \item 基于开源文献管理软件zotero,易于公开共享
\end{itemize}



\ResumeItem{科研菜鸟营——海洋与气象}
[ 学术社群 ]
[2021.09 — 至今]
\begin{itemize}
  \item 旨在帮助海洋及气象相关学科的研究生进行学术基础培训
  \item 分享人工智能算法和生成式人工智能使用经验
  \item 分享数据下载,数据处理策略
\end{itemize}

\ResumeItem{新云天气象海洋科技服务工作室}
[ 创新创业团队 ]
[2020.01 — 至今]
\begin{itemize}
 \item 参与组建该团队,为骨干成员。帮助该团队建立了早期的新媒体信息发布渠道
  \item 该工作室以“新云天气象”公众号为平台,立足广东海洋大学,统筹南京信息工程大学、中国海洋大学和兰州大学等众多高校资源,为气象行业提供新媒体信息发布、会议展览、人才交流和国际出版等服务。
  \item 该工作室相关微信公众号在大气科学专业学生中具有较高知名度
\end{itemize}


\ResumeItem{“大书虫”志愿服务团队}
[ 志愿服务项目 ]
[2018.04 — 2018.04]
\begin{itemize}
  \item 旨在推广普及对大学图书馆、全国各省图书馆的数据库资源的充分使用
  \item 团队获得“学雷锋月”志愿服务优秀团队荣誉
  \item 个人获得“学雷锋月”志愿服务优秀队长荣誉
\end{itemize}

\ResumeItem{印度尼西亚海域海洋热浪变化特征及动力分析}
[硕士学位论文]
[2022.12—2024.03] 

\begin{itemize}
  \item 利用印度尼西亚海域的SST数据进行海洋热浪的监测,得出了该海域海洋热浪的一般特征
  \item 利用高分辨海洋再分析数据对研究海域的海洋热浪动力机制进行了探究,得到了印度尼西亚贯穿流和海气净热通量各自具体贡献
\item 利用多种气候模态分别对正负相位下的海洋热浪特征进行合成分析,结果发现研究海域明显的遥相关特性,这与以往的认知与很大差异
\end{itemize}

\ResumeItem{基于深度学习的波浪预报技术研究}
[本科学位论文]

\begin{itemize}
  \item 分别利用LSTM算法和EMD-LSTM融合算法对美属波多黎各的波浪波高数据进行预报,并将预报结果与实际结果进行对比。结果发现EMD-LSTM融合算法具有更好的预测质量
\end{itemize}

\section[管理经验]{管理经验}
\ResumeItem{哈尔滨工程大学\ 水声工程学院}
[21级海洋科学联合培养研究生\ 副联络人]
[2022.09—2024.06] 

\begin{itemize}
  \item \textbf{学位材料管理。}在主联络人出国期间,负责了自然资源部第一海洋研究所与哈尔滨工程大学联合培养研究生的开题答辩、中期答辩、毕业答辩学位材料的邮寄工作
    \item \textbf{联络本部与青岛基地。}在主联络人出国期间,负责了和哈尔滨工程大学水声工程学院及哈尔滨工程大学青岛创新发展基地的党团关系联络工作
\end{itemize}

\ResumeItem{广东海洋大学\ 海洋与气象学院}
[17级物理海洋班\ 班长]
[2017.09—2018.06] 

\begin{itemize}
  \item \textbf{事务管理。}对班级同学的信息进行了建档管理,基于该建档完成了多次学院下发的表格填报任务,有效减轻了同学们的事务负担
  \item \textbf{班级服务。}经常来往于同学宿舍分享课程资料,软件及代码,与同学们建立了良好关系
\end{itemize}

\ResumeItem{广东海洋大学\ 学生社团}
[蓝丝带海洋保护协会\ 办公室部长]
[2018.09—2019.06] 

\begin{itemize}
  \item \textbf{组织净滩志愿服务活动。}组织实施了“龙海天”净滩志愿服务活动,对海滩垃圾进行统计,与社会上的环保组织进行资料共享,并负责了志愿者的服务计时打卡
    \item \textbf{社团档案管理。}对团队历年的照片、视频、海报、人事信息、年报等历史档案进行收纳整理,并参与完成了社团年报
\end{itemize}

\ResumeItem{广东海洋大学\ 海洋与气象学院}
[17级物理海洋班\ 学习委员]
[2019.09—2020.06] 

\begin{itemize}
  \item \textbf{学工事务。}将每周专业课程课后作业进行统计整理
    \item \textbf{升学情报收集。}将各所高校的研究生招生信息进行收录整理
    \item  \textbf{班级班报设计制作。}将课程作业信息和升学就业信息制作成班级班报,委托系主任进行审核和打印,每周到各个宿舍进行发放
\end{itemize}


\section[资格证书]{资格证书}
\begin{itemize}
  \item {大学英语六级CET-6}
  \item {国家信息安全水平证书(一级)NISP-1}
\end{itemize}

\section[荣誉奖项]{荣誉奖项}
\begin{itemize}
  \item {2023年12月获得哈尔滨工程大学研究生二等学业奖学金}
  \item {2022年12月获得哈尔滨工程大学研究生二等学业奖学金}
  \item {2021年12月获得哈尔滨工程大学研究生三等学业奖学金}
  \item {2020年12月获得广东海洋大学三等奖学金}
  \item {2019年12月获得广东海洋大学三等奖学金}
  \item {2018年12月获得广东海洋大学二等奖学金}
\end{itemize}

\section[技术背景]{技术背景}
\begin{itemize}
  \item \textbf{编程语言}: 常用 Python和MATLAB,数据分析处理经验丰富; 熟练掌握FORTRAN;有C语言基础。
  \item \textbf{操作系统}: 熟练掌握Windows, Linux, MacOS操作系统.
    \item \textbf{办公软件}: 熟练掌握Word, Excel, PowerPoint, Outlook.
  \item \textbf{其他}: 有云计算和容器化技术的实践经验。
\end{itemize}



\section{个人总结}

\begin{itemize}
  \item 人际关系方面,乐于助人,社会关系较好
\item 实践能力方面,编程经验丰富,网络技术和计算机应用能力强
\item 认知层次方面,重视调研,善于情报分析,知识储备深厚,视野广阔
\item 学生品质方面,勤奋好学,学习认真扎实
\item 工作作风方面,严谨踏实,工作认真负责
\end{itemize}


\end{document}
